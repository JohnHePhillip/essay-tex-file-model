\documentclass[11pt]{article}
\usepackage[T1]{fontenc}
\usepackage{fixltx2e}
\usepackage{graphicx}
\usepackage{longtable}
\usepackage{float}
\usepackage{wrapfig}
\usepackage{soul}
\usepackage{textcomp}
\usepackage{marvosym}
\usepackage{wasysym}
\usepackage{latexsym}
\usepackage{savesym}
\savesymbol{iint}
\savesymbol{iiint}
\usepackage{amsmath}
\usepackage{ctex}
\usepackage{amssymb}
\usepackage{hyperref}
\usepackage{indentfirst}
\usepackage{keyval, xcolor, calc,ifthen}
\tolerance=1000
\usepackage{listings}
\hypersetup{CJKbookmarks=true}
\providecommand{\alert}[1]{\textbf{#1}}


\definecolor{mygreen}{rgb}{0,0.6,0}  
\definecolor{mygray}{rgb}{0.5,0.5,0.5}  
\definecolor{mymauve}{rgb}{0.58,0,0.82}  
  

\lstset{ %  
  backgroundcolor=\color{white},   % choose the background color; you must add \usepackage{color} or \usepackage{xcolor}  
  basicstyle=\footnotesize,        % the size of the fonts that are used for the code  
  breakatwhitespace=false,         % sets if automatic breaks should only happen at whitespace  
  breaklines=true,                 % sets automatic line breaking  
  captionpos=bl,                    % sets the caption-position to bottom  
  commentstyle=\color{mygreen},    % comment style  
  deletekeywords={...},            % if you want to delete keywords from the given language  
  escapeinside={\%*}{*)},          % if you want to add LaTeX within your code  
  extendedchars=true,              % lets you use non-ASCII characters; for 8-bits encodings only, does not work with UTF-8  
  frame=single,                    % adds a frame around the code  
  keepspaces=true,                 % keeps spaces in text, useful for keeping indentation of code (possibly needs columns=flexible)  
  keywordstyle=\color{blue},       % keyword style  
  %language=Python,                 % the language of the code  
  morekeywords={*,...},            % if you want to add more keywords to the set  
  numbers=left,                    % where to put the line-numbers; possible values are (none, left, right)  
  numbersep=5pt,                   % how far the line-numbers are from the code  
  numberstyle=\tiny\color{mygray}, % the style that is used for the line-numbers  
  rulecolor=\color{black},         % if not set, the frame-color may be changed on line-breaks within not-black text (e.g. comments (green here))  
  showspaces=false,                % show spaces everywhere adding particular underscores; it overrides 'showstringspaces'  
  showstringspaces=false,          % underline spaces within strings only  
  showtabs=false,                  % show tabs within strings adding particular underscores  
  stepnumber=1,                    % the step between two line-numbers. If it's 1, each line will be numbered  
  stringstyle=\color{orange},     % string literal style  
  tabsize=2,                       % sets default tabsize to 2 spaces  
  %title=myPython.py                   % show the filename of files included with \lstinputlisting; also try caption instead of title  
}  



\title{Homework3}
\author{计科80 杨乾澜}
\date{\today}
\hypersetup{
  pdfkeywords={},
  pdfsubject={},
  pdfcreator={Emacs Org-mode version 7.9.3f}}



\begin{document}

\maketitle

\section{Chapter 12}

\subsection{Problem 5}

\begin{lstlisting}
clear Z
incr X
while X not 0:
    clear temp1
    while Y not 0:
        decr Y
        incr temp1
    while temp1 not 0:
        decr temp1
        incr Y
        decr X
    incr Z
while Z not 0:
    decr Z
    incr X
decr X
\end{lstlisting}

X represents the final answer.

\subsection{Problem 15}


\newcommand{\tabincell}[2]{\begin{tabular}{@{}#1@{}}#2\end{tabular}}

\begin{tabular}{|c|c|c|c|c|}% 通过添加 | 来表示是否需要绘制竖线
\hline  % 在表格最上方绘制横线
Current state & \tabincell{c}{Current cell \\ content} & Value to write & Direction to move &\tabincell{c}{ New state \\ to enter}\\
\hline
START & * & * & Left & PACK\\
\hline
PACK  & 0 & * & Right & MOVE0\\
\hline
PACK & 1 & * & Right & MOVE1\\
\hline
MOVE0 & 0 & 0 & Right & MOVE0 \\
\hline
MOVE0 & 1 & 1 & Right & MOVE0 \\
\hline
MOVE1 & 0 & 0 & Right & MOVE1 \\
\hline
MOVE1 & 1 & 1 & Right & MOVE1 \\
\hline
MOVE0 & * & * & Left & WRITE0 \\
\hline
MOVE1 & * & * & Left & WRITE1 \\
\hline
WRITE0 & 1 & 0 & Left & WRITE1 \\
\hline
WRITE0 & 0 & 0 & Left & WRITE0 \\
\hline
WRITE1 & 1 & 1 & Left & WRITE1 \\
\hline
WRITE1 & 0 & 1 & Left & WRITE0 \\
\hline
WRITE0 & * & 0 & Left & PACK \\
\hline
WRITE1 & * & 1 & Right & PACK\\
\hline
PACK & * & * & No move & HALT\\

\hline % 在表格最下方绘制横线
\end{tabular}

\subsection{Problem 22}

Approach 1: Ask him/her Which number is equal to one plus one. If he/she say two then he/she is a truth teller, otherwise he/she is a liar.

Approach 2: Ask him/her to write down the answer of "Which person are you, a truth teller or a liar?" on a paper, then tell you the words written on it. As a result, what he told you is the identity of him.

\subsection{Problem 24}

The halting problem shows that there are some problems that couldn't be solved by machines, which means the capabilities of machines is limited.

\subsection{Problem 25}

During we asking all the people, there exists at least people answering that his/her birthday is on that day.

After we asking all the people, there wasn't any people answering that his/her birthday is on that day.

We can enumerate all the positive numbers from 0 to bigger, if there exists some number we wanted, we will find it eventually. However, the time of this algorithm may be infinite.

I think we can't solve this problem.

\subsection{Problem 32}



\subsection{Problem 35}

``Select three numbers between 1 and 100." is nondetermin.

``Select one of the chosen numbers and give that number as the answer" is nondetermin.

\subsection{Problem 36}

The algorithm has a poly-nomial complexity.Because "Pick a collection of numbers" will take at last $O(n)$ time and check if the sum of these numbers is equal to 125 takes also $O(n)$ time. So the total complexity of the algorithm is $O(n)$.

\subsection{Problem 43}

Polynomial problems: Sorting an list of numbers
Nonpolynomial problems: Travelling salesman problem
Unsolvable problems: The Halting problem

\subsection{Problem 45}

The answer is "257, 771, 391, 304".

My algorithm is to enumerate all the subset of the list and check if their sum is equal to 1723. The complexity is $O(2^n *n )$.

\subsection{Problem 50}

The private key is 43.

In this case, n is small enough to find $p=7$ and $q=11$ so that $p*q = n$ in a $O(n)$ time. After that, we can find the private key d which satisfys $d*e = k(p-1)(q-1) +1$ with a extended-Euclidean Algorithm. 

\section{Chapter 6}

\subsection{Problem 4}

Suppose the variable X is tored in the memory cell whose address is 1A, while Y is 1B, and Z 1C, W 1D.

\begin{lstlisting}
2000
111A
121B
131D
B10E
5412
B010
5423
341C
C000
\end{lstlisting}

\subsection{Problem 22}

\begin{lstlisting}
CASE W IS
  WHEN 5=> Z:=7;
  WHEN 6=> Y:=7;
  WHEN 7=> X:=7;
END CASE;
\end{lstlisting}

\subsection{Problem 33}

This sentence can represent X = (3+2) *5 or X = 3+(2*5),either.

\subsection{Problem 44}


\begin{figure}[h]%%图
	\includegraphics[width=0.4\linewidth]{phrase_tree2.jpg}  %插入的图,包括JPG,PNG,PDF,EPS等,放在源文件目录下
\end{figure}

\subsection{Problem 45}

First of all, we can load 2 to a register. Then we can predict that X equals to 5 so that we can let compute Z by X + 2. If X doesn't equal to 5, we can let Z be Z + 2 then. In this way, we only need to load 2 to the register.

\section{Bonus problem}

\end{document} 
